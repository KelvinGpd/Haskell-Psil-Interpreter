\documentclass{article}
\usepackage{graphicx} % Required for inserting images

\title{Rapport Haskell-Interpréteur Psil}
\author{Samuel Maltais et Kelvin Chen}
\date{3 Juin 2023}

\begin{document}

\maketitle

\section*{Problèmes rencontrés:}
\begin{enumerate}
    \item s2l transformation de let en Llet et de hastype:
    vu que nous avons défini dans notre code s2l le cas pour un appel de fonction qui utilise un Ssym quelconque, "let", qui est aussi un Ssym se faisait traiter par le mauvais cas d'un appel de fonction. Ceci nous induisait en erreur dans nos premiers tests dans lesquels on passait des fonctions let dans lexpOf. Haskell traite les fonctions non par ordre de spécificité, mais plutôt par l'ordre dans lequel il apparaissent. Ainsi, il fallait définir les cas pour let et pour hastype avant celui de l'appel de fonction. 
    
    \item Syntaxe de haskell, comment utiliser un Vop:
    À un point, nous n'arrivions pas a utiliser les fonctions définies comme des Vop. C'était pourtant simple, un Vop contient une fonction lambda qui retourne ou non une fonction lambda. Comment extraire cette fonction et l'appeler ? Nous étions completement perdus !! Nous l'avons finalement résout en faisant un case si notre variable func avait la forme Vop (fonctionQuelconque), puis en appelant cette fonction sur l'argument, ça a fonctionné
    
    \item Hashtype et sa signification
    Le problème ici résidait simplement dans la signification de hashtype. Est ce qu'on le considère dans process decl comme une declaration ? Non, car hashtype peut etre associé a n'importe quoi, pas seulement une variable. Notre systeme infère deja aussi les types, nous avons alors résolus que hashtype devait servir de verification, pour faire en sorte que le programmeur soit certain du type de ses arguments.
    
    \item ajouter des :

    
    
    
    \item Appel de fonctions et currying:
    En premier lieu, le problème le plus flagrant est de handle la création de fonction
     \item élaboration des tests et la fonction "run": 
     nous n'étions pas capable d'appeler la fonction run avec le path de notre fichier texte contenant les tests. Après plusieurs essais, une recherche en ligne nous a dévoilé qu'il fallait spécifier le path avec les doubles backslash, vu que haskell interprète "\" dans les chaines comme un charactère d'échappement. 
\end{enumerate}

\section*{Surprises:}
\begin{enumerate}
    \item Appel de fonctions et currying
    En premier lieu, le problème le plus flagrant est de handle la création de fonction

    \item Snil et son utilisation dans les Sexp:
    à plusieurs reprises, les erreurs de nos tests reliaient aux fonctions s2t et s2l, qui convertissait un Sexp en le sa L equivalent. Ce n'est qu'en testant avec la fonction SexpOf suivi de l'expression que nous nous sommes rendus compte que plusieurs Ssym étaient précédés d'un Snil, afin d'isoler le symbole dans le Scons. Par exemple, pour (Int Int -> Int), il n'y a rien qui nous indiquait que le premier Ssym Int devait être précéde de Nil. Pour contrer cette surprise, nous avons introduit au cas de base: s2t (Scons Snil (Ssym 'Int')). 

    \item
    

\end{enumerate}

\section*{Choix que nous avons fait:}
\begin{enumerate}
    \item Appel de fonctions et currying
    En premier lieu, le problème le plus flagrant est de handle la création de fonction anonymes. Mettons (+ 2) 3, une premiere étape est de créer la fonction intermediaire (+ 2). Pour cela, nous avons simplement résolu d'avoir une chaine d'appel lorsque le programme est présenté devant un Scons( exp1 exp2) qui ne match pas d'autre cas. Il fera allors un appel de (+ 2) sur 3 mais evalue (+ 2) en premier. Il va donc aller en profondeur, jusqu'a ce qu'il trouve un Syym, la fonction déja définie avec laquelle nous créons des fonctions intermediaire. C'étais aussi un aspect surprenent de ne pas avoir besoin de faire des fonctions anonymes pour des tels cas, puisque nous allons en profondeur d'abord.
    \item Valeure de Recursive et call by name


    Premièrement nous avons remarqués que dans process declaration l'environement est changé seulement après avoir evalué l'expression. Nous savions donc que trouver la valeure de Recursive ne serait pas possible. Nous devions donc faire un call by name, pour que la valeure de Recursive ne soit jamais recherchée. Le problème que nous avons toujours pas résulu est comment delay l'évaluation de recursive. La solution est probablement de faire en sort que eval puisse prendre différentre types de inputs que des values. Nous avons fait ce travail derniere minute, donc n'avons pas réalisés comment implémenter un call by name.
    

    
\end{enumerate}

\section*{Ce qu'on aurait fait différemment}
\begin{enumerate}
    \item Suivre les conseils vis a vis les test. Nous avons ecrit beaucoup de code d'un seul coup, et donc avons du ajuster au fur et à mesure lorsqu'il était évident que les tests ne produisaient pas les résultats désirés. En effet, nous avions rédigé plus que le trois quart du code avant d'avoir passé à un seul test.
    \item Opter pour une structure de données plus efficace que celle offerte dans le code de base. Il nous aurait été possible de tranformer les Sexp en [a] une liste séquentielle d'arguments, ou a peut etre un Int ou un Ssym. Cela aurait été de loin plus efficace que l'algorithme récursive (en profondeur) que nous employons pour évaluer les Sexp. En effet, on aurait pu n'avoir qu'une seule fonction avec un cas de base d'une liste vide et puis passer un par un les arguments. Bien que cela prendrait plus de temps car il faudrait réecrire des fonctions du code de base, cela aurait été notre choix afin d'éviter la longuer de code ainsi que les évaluations de plusieurs cas (à chaque fois > 5 cas) 
\end{enumerate}

\section*{Ce que nous avons appris}
\begin{enumerate}
    \item Nous avons appris à s'adapter à du code complexe sans documentation. C'est quaisiment en sorte du reverse engineering. Bien que le code était bien commenté, nous étions impatient de nous même ecrire, et donc nous réalisions certaines choses trop tard. 

    \item Nous avons appris comment fonctionnaient les parsers, et comment les compilateurs
    
\end{enumerate}


\end{document}
